\documentclass[]{book}

%These tell TeX which packages to use.
\usepackage{array,epsfig}
\usepackage{amsmath}
\usepackage{amsfonts}
\usepackage{amssymb}
\usepackage{amsxtra}
\usepackage{amsthm}
\usepackage{mathrsfs}
\usepackage{color}

%Here I define some theorem styles and shortcut commands for symbols I use often
\theoremstyle{definition}
\newtheorem{defn}{Definition}
\newtheorem{thm}{Theorem}
\newtheorem{cor}{Corollary}
\newtheorem*{rmk}{Remark}
\newtheorem{lem}{Lemma}
\newtheorem*{joke}{Joke}
\newtheorem{ex}{Example}
\newtheorem*{soln}{Solution}
\newtheorem{prop}{Proposition}

\newcommand{\lra}{\longrightarrow}
\newcommand{\ra}{\rightarrow}
\newcommand{\surj}{\twoheadrightarrow}
\newcommand{\graph}{\mathrm{graph}}
\newcommand{\bb}[1]{\mathbb{#1}}
\newcommand{\Z}{\bb{Z}}
\newcommand{\Q}{\bb{Q}}
\newcommand{\R}{\bb{R}}
\newcommand{\C}{\bb{C}}
\newcommand{\N}{\bb{N}}
\newcommand{\M}{\mathbf{M}}
\newcommand{\m}{\mathbf{m}}
\newcommand{\MM}{\mathscr{M}}
\newcommand{\HH}{\mathscr{H}}
\newcommand{\Om}{\Omega}
\newcommand{\Ho}{\in\HH(\Om)}
\newcommand{\bd}{\partial}
\newcommand{\del}{\partial}
\newcommand{\bardel}{\overline\partial}
\newcommand{\textdf}[1]{\textbf{\textsf{#1}}\index{#1}}
\newcommand{\img}{\mathrm{img}}
\newcommand{\ip}[2]{\left\langle{#1},{#2}\right\rangle}
\newcommand{\inter}[1]{\mathrm{int}{#1}}
\newcommand{\exter}[1]{\mathrm{ext}{#1}}
\newcommand{\cl}[1]{\mathrm{cl}{#1}}
\newcommand{\ds}{\displaystyle}
\newcommand{\vol}{\mathrm{vol}}
\newcommand{\cnt}{\mathrm{ct}}
\newcommand{\osc}{\mathrm{osc}}
\newcommand{\LL}{\mathbf{L}}
\newcommand{\UU}{\mathbf{U}}
\newcommand{\support}{\mathrm{support}}
\newcommand{\AND}{\;\wedge\;}
\newcommand{\OR}{\;\vee\;}
\newcommand{\Oset}{\varnothing}
\newcommand{\st}{\ni}
\newcommand{\wh}{\widehat}

%Pagination stuff.
\setlength{\topmargin}{-.3 in}
\setlength{\oddsidemargin}{0in}
\setlength{\evensidemargin}{0in}
\setlength{\textheight}{9.in}
\setlength{\textwidth}{6.5in}
\pagestyle{empty}



\begin{document}


\begin{center}
{\Large CWI INRIA Meeting \ January 2018}\\
\textbf{Mandar Chandorkar}, \textbf{Enrico Camporeale}, \textbf{Mich\'el\`e Sebag}, \textbf{Cyril Furtlehner}, \textbf{Giancarlo Fissore}
\end{center}

\vspace{0.2 cm}

\section{Solar Wind Prediction}

The general aim is to predict key L1 quantities (solar wind speed; $v_{sw}$ and north-south component of interplanetary magnetic field; $B_z$) from solar images and auxillary data. 

\subsection{Committee of Experts: Meta-Model}\label{model1}

Consider the task of predicting $v_{sw}(t)$ from stacked solar images $S(t) = [S(t:t-24), S(t-24:t-48), S(t-48:t-72)]$, we can train CNN based predictive models $\mathcal{M}_1[\hat{v}_{sw}(t) = f_{1}(S(t:t-24))]$, $\mathcal{M}_2[\hat{v}_{sw}(t) = f_{2}(S(t-24:t-48))]$ and $\mathcal{M}_3[\hat{v}_{sw}(t) = f_{3}(S(t-48:t-72))]$.

The next task is then to train a meta model $\mathcal{F}[\hat{v}_{sw}(t) = \sum_{i}{\alpha_{i}(S(t))f_{i}(S_{i}(t))}]$, where $\alpha_{i}(S(t))$ is the confidence $\mathcal{F}$ places on model $\mathcal{M}_i$. We can also consider the more general model $\mathcal{F}[\hat{v}_{sw}(t) = f(S(t), f_{i}(S_{i}(t)))]$, after training the committee constituents, the meta-model can be represented and trained as a neural architecture.

\textbf{Difficulty Level}: Moderate, can be achieved using standard neural network toolboxes. 

\subsection{CNN \& Hybrid 1d MHD}

In this approach, we use a CNN architecture to predict a latent variable $v_{sw}(r = 30r_s, \phi)$ and propagate it forward with a simplified 1-D \textit{magneto-hydro dynamics} model which can be expressed as follows.

\begin{equation}
    v_{i+1,j} = v_{i,j} + \frac{\Delta r \Omega_{rot}}{v_{i,j}} \left (\frac{v_{i,j+1} - v_{i,j}}{\Delta \phi} \right )
\end{equation}

The 1-D MHD model can then be integrated to provide an estimate of the propagation time of the solar wind disturbances. In order to perform inference, we must cast the problem as a latent variable EM method. 

\textbf{Difficulty Level}: Moderate, greater than method \ref{model1}, will take more time for training due to latent variables and convergence time of EM procedures. Number of latent variables depends on the coarseness of grid in the Carrington longitude $\phi$, this will generally have the effect of larger training times and identifiability (multiple initial velocity profiles $v_{sw}(r = 30r_s, \phi)$ might yield similar values of L1 solar wind).

\subsection{CNN based prediction of time delay and solar wind}

A CNN based architecture which predicts simultaneously $\Delta t$ and $v_{sw}(t + \Delta t)$ from a set of solar images $S(t)$.

\textbf{Challenges, Issues}: \begin{enumerate} \item Is this problem supervised/unsupervised/semi-supervised ? \item How will the training data be pre-processed to enable model training ? \item Need careful thought on how an appropriate loss function can be formulated for this task \end{enumerate}


\end{document}


